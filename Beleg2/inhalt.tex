%%======================================================================
%% Haupttext
%%======================================================================

\renewcommand{\baselinestretch}{1.2}\normalsize

\chapter{INTRODUCTION}
\label{sec:introduction}

\pagenumbering{arabic}
%\setcounter{page}{2}
\setcounter{footnote}{1}

Write something about IT Service Management, its goals and why software supports achieving these goals. Also describe goals of your project here - why are we doing this? (About one page in sum.)

\chapter{Evaluation of General Characteristics of the Software}

Write a few introductory lines.

\section{Basis facts about the software solution}

Please describe the basic characteristics of the software. Is is a web-based solution? Is it installable or just available as a service? How much effort does the installation take (describe and give a time for the installation)? How old is the software? When did its development start? Which version is it (0.1?)? Is it well established? How many customers are using it? Is it still further developed - are new releases planned and when was the last release? 

Please describe the basic characteristics of the software. 

Is is a web-based solution?

Ja

Is it installable or just available as a service? 

Die Software ist installierbar.



How much effort does the installation take (describe and give a time for the installation)?

Die folgenden Schritte werden auf folgendem Testrechner ausgef�hrt:

Ubuntu 12.04

Es wird die Free Trial Version von Nagios XI heruntergeladen unter:

http://www.nagios.com/products/nagiosxi

Die Trial Version ist eine 60 Tage Testversion von Nagios XI.

Es wird die VMware Virtual Machine (64-bit) Version 2012R1.6 heruntergeladen unter:

http://library.nagios.com/library/products/nagiosxi/downloads/main

Auf der viruellen Maschine l�uft CentOS 6.x und Nagios XI 2012 ist installiert.

Um die VMware Virtual Machine starten zu k�nnen, muss VMware installiert werden.
VMware kann kostenfrei heruntergeladen werden unter:

\url{https://my.vmware.com/web/vmware/free#desktop_end_user_computing/vmware_player/5_0}

Der heruntergeladene VMware-Player wird installiert mit:

sudo sh VMware-Player-5.0.1-894247.x86\_64.bundle

Der VMware-Player wird gestartet und es wird die virtuelle Maschine ausgew�hlt mit 

Open a virtual Maschine.

Die ausgew�hlte virtuelle Maschine wird gestartet mit 

play virtual maschine.

Die virtuelle Maschine wird gestartet. CentOS wird gebootet und Nagios gestartet.

Auf Nagios XI kann jetzt im Browser zugegriffen werde unter:

http://192.168.0.100/

\begin{figure}[htp]
\centering
\includegraphics[width=0.6\textwidth]{ingo/bilder/Startseite}
\caption{Startseite von Nagios}
\label{fig:StartseiteVonNagios}
\end{figure}

Das Default Root Passwort ist nagiosxi


How old is the software? When did its development start? 

1999 ver�ffentlichte Ethan Galstad Nagios - dass damals noch NetSaint hie� - als Open Source Projekt.
http://www.nagios.org/about/history

Which version is it (0.1?)? 

Netsaint 0.0.1
\url{http://www.ussrback.com/UNIX/audit/netsaint/index.html}

Is it well established? How many customers are using it? 

Es wird gesch�tzt, dass es weltweit ca 1 Million Nagios Nutzer gibt.
http://www.nagios.org/about/community

Is it still further developed - are new releases planned and when was the last release?

Es wird immernoch weiterentwickelt. Neue Releases sind geplant. Die letzte Version 2012R1.6 ist von 15. Februar 2013.

\section{Versions and price}

Please describe versions, prices and cost models of the software. Is the software open source?
\input{ingo/VersionsAndPrice}

\section{Usability}

Does the user interface and layout look up to date? Is it easy to understand and learn the functionalities of the software? Is the user interface efficient? Are the functionalities intuitive? Is the choice of the colors reasonable? Please answer the questions and grade the usability of the software - 5-extraordinary good - 1-very poor.

\section{Performance}

Describe if the performance of the software. Is it satisfying? Are the response times in order? Are there any performance critical operations? Please answer the questions and evaluate the performance between 5-extraordinary good and 1-the screen war frozen with the first klick :)

\section{Documentation}

Describe how well the software is documented on a scale between 5-extraordinary well to 1-no documentation available. Please also describe which tutorials, seminars or certification courses are available. 
\section{Support}

How is the software supported? Is there a number to call, a web form/ email address, a forum (and if so, how are the replies - fast and well explaining?)? Does the support have to be payed? Try to ask the support a question and report the reaction. If you got a reply - was the reply competent? Evaluate the support between 5-extraordinary good to 1-no support.

Mit einer Nagios XI Lizenz enth�lt:

\begin{itemize}
 \item technischen Support - via EMail oder in einem speziellen customer-only Abschnitt im online forum unter

\url{http://support.nagios.com/forum}

Je nach Lizenz-Level werden pro Jahr bis zu 10 Support Incidents bearbeitet.

\item Training - ein volles Jahr Zugriff auf Ressourcen f�r das Selbststudium von Nagios XI und Nagios addons.

\item eine unbefristete Lizenz - die Lizenz der Software ist dauerhaft, selbst wenn kein zuk�nftiger Support oder Wartungsvertr�ge geschlossen werden.

\item die Nagios Library - f�r ein volles Jahr kann auf spezielle Nagios Libraries zugegriffen werden, mit customer-only Tutorials, Videos und Tech Tipps.

\item Produkt-Einfluss - Feature Requests k�nnen eingereicht werden.

\item freie Lizenzwahl f�r selbstgeschriebene Erweiterungen - es k�nnen beliebige Lizenzen f�r eigene Erweiterungen f�r Nagios XI gew�hlt werden: z.B. open source, propriet�r oder public domain.
\end{itemize}

Die Leistungen und die Preise f�r den Support richten sich nach dem Lizenz-Level, dass der Kunde gekauft hat. Die Lizenzlevel sind in der Tabelle Lizenzleel auf Seite \pageref{LizenzLevel} aufgef�hrt.

Ein Eintrag im Forum wird innerhalb von 24 Stunden beantwortet.


\section{Errors and Bugs}

Describe errors and bugs of the software which you found during testing.

\chapter{Evaluation of the ITSM Specific Functionalities}

\section{ITSM Processes}
\label{sec:itsmprocesses}

Describe the support of the processes with a number from 1-5 (5 for fully supported), describe also "what" it supported and "what not". Use our scribe notes with detailed information about the processes to have an idea about what could be supported. Give further descriptions "how" it is supported and about the usability of the process features in the column "comments". Describe as many details as possible.

\begin{table}[h!]
\caption{Supported ITSM Processes}
\vspace*{0.3cm}
\begin{tabular}{|p{6.1cm}|p{4.5cm}|p{4.5cm}|}\hline
\textbf{ITSM Process}             &\textbf{Supported (1-5)}        &\textbf{Comments}\\\hline\hline
\textbf{Service Strategy}          &                                &\\
Strategy Generation            &                                &\\
Demand Management                   &                                &\\
Service Portfolio Mgmt.                   &                                &\\
Financial Management                   &                                &\\\hline
\textbf{Service Design}          &                                &\\
Service Catalogue Mgmt.                   &                                &\\
Service Level Management                   &                                &\\
Capacity Management                   &                                &\\
Availability Management                   &                                &\\
IT Service Continuity Mgmt.                   &                                &\\
Information Security Mgmt.                   &                                &\\
Supplier Management                   &                                &\\\hline
\textbf{Service Transition}          &                                &\\
Transition Planning and Support                   &                                &\\
Change Management                   &                                &\\
Service Asset and Configuration Mgmt.                   &                                &\\
Release and Deployment Mgmt.                   &                                &\\
Service Validation and Testing                   &                                &\\
Evaluation                   &                                &\\
Knowledge Management                   &                                &\\\hline
\textbf{Service Operation}          &                                &\\
Event Management                   &                                &\\
Incident Management                   &                                &\\
Request Management                   &                                &\\
Problem Management                   &                                &\\
Access Management                   &                                &\\\hline
\textbf{Continual Service Improvement}          &                                &\\
7-Step Improvement                   &                                &\\
Service Reporting                   &                                &\\
Service Measurement                 &                                &\\\hline
\textbf{General}                   &                                 &\\
Service Desk                       &                                 &\\
Raci Authority Matrix                    &                           &\\\hline
\end{tabular}
\label{tab:SupportedITSMProcesses}
\end{table}

\section{IT Service Management Roles}
\label{sec:ITSMRoles}

Described the support of the ITSM roles individually.

\section{Scenarios}
\label{sec:scenarios}

\subsection{Design of an Email Service}
\label{sec:emailServiceDesign}

Design an email service as a new service within the service management software which has to be evaluated. Please consider if supported:
\begin{itemize}
\item Creating and formalizing a \emph{Service Design Package} (SDP)
\item Creating and formalizing \emph{Service Acceptance Criteria} (SAC)
\item Creating and formalizing \emph{Service Level Requirements} (SLR) which develop to the \emph{Service Level Agreement} (SLA) within the design process.
\end{itemize}

Lookup all necessary information about the documents in your scribe notes. Are measurable goals for the services supported by the software? How about SLA-Monitoring (SLAM) and Service Improvement Plans? Are different service levels for the same service supported? Are different SLA types supported (service-based, customer based, multi-level)?

Describe how your software supports this procedure. Use also screen shots to visualize that.

\subsection{ITIL Roles}
\label{sec:itilRoles}

Check which of the ITIL roles (\url{http://wiki.en.it-processmaps.com/index.php/Roles_within_ITIL_V3}) are supported by your software or how it is possible to define ITIL roles for your service. Then establish as many ITIL roles as possible in your test system.

\subsection{Service Catalogue}
\label{sec:serviceCatalogue}

Check if the life cycle phases \emph{Service Pipeline}, \emph{Service Catalogue} and \emph{Retired Services} are supported by your software. Add the new Email Service as new service to the service pipeline, then transfer it to the service catalogue and finally retire the service. Is it possible to get information about all services in the life cycle? If yes describe how that works. Which information are available for each service? Is there a division in \emph{business service catalogue} and \emph{technical service catalogue}?

Describe how your software supports this procedure. Use also screen shots to visualize that.

\subsection{Further Management of the Email Service}
\label{sec:furtherManagement}

Note that you need to preserve capacities for your email service. Is there a capacity management information system? If yes then develop a capacity plan and  reserve capacities for your email service. How about planning of personnel resources?

Also develop an availability plan for your email service if this is supported by the software. Is the ITIL role of the Availability Manager supported?

Is it supported by the software to plan capacities and availability in an alternating way?

How about a continuity plan for your Email service - is it possible to establish that in the software? Develop an exemplary continuity plan for the email service.

How is information security management and supplier management supported. If supplier management is supported, add exemplary suppliers with underpinning contracts for basis functionality of your email service - e.g. use cloud services to store the emails or your service. Safe the contract in the supplier and contract data base if available (SCD - Supplier and Contract Database).

\subsection{Configuration Management}
\label{sec:configurationManagement}

Add all necessary configuration items to implement your email service as configuration items in the Configuration Management Database (CMDB) of the system. Note that it is normal to have thousands if not millions of configuration items within the system. This is of course not the case in our test system but consider adding different CI-types as necessary infrastructure and servers, routers, documents, chairs, IP-adresses, other hardware and services, buildings, persons, etc. Add for each of the items a configuration record with all describing information. Are agents supported, which write infrastructure automatically in the CMDB? Is a CMS available to host the CMDB?

\subsection{Change Management}
\label{sec:changeManagement}

Simulate all three types of changes which you know from the lecture. Rise a RFC (request for change) first. Then initiate a normal change to your service, e.g. migrating the email service to a new server. Also increase the storage capacity of your email service from the initial value of $x$ MB to $y$ MB. Since this change is not very risky, establish it as \emph{standard change}. Define the CAB (Change Advisory Board) and a ECAB (Emergency Change Advisory Board). Simulate the situation that your email system got hacked - how is the handling of that situation supported by your software?

Now think about developing your email service further, e.g. by adding IMAP or POP or by adding a web interface. Plan those changes in a \emph{Forward Schedule of Changes} (FSC) if available. Are the 7Rs supported somehow to make sure that the right questions are asked before the change is realized?

\subsection{Definite Media Library}
\label{sec:definiteMediaLibrary}

Is there a \emph{Definite Media Library} (DML) to represent media CIs in different versions? All media items should also be available as CI in the CMDB. Think about which software you need to realize your email service and create configuration items with Master copies of this software which are also represented in the DML.

\subsection{Knowledge Management}
\label{sec:knowledgeManagement}

Consider to safe further information about your email service, as current users, status information about the service utilization, down times, etc. Safe those information to the \emph{Service Knowledge Management System} (SKMS) if possible. Is it supported to safe status information about the service automatically?

\subsection{Event, Incident and Problem Management}
\label{sec:eventIncidentProblemManagement}

Since we did not consider Service Operation in the lecture so far, we need to have a brief introduction. Note that events can be any change of status - informative events where no action is required, hitting some predefined level for some parameter, exceptions that some resource or service is not in operation as usual. Incidents are events which interrupt or potentially interrupt a service. Events which are incidents have to be reported using an interface of the \emph{Event Management System} to the \emph{Incident Management}. Also the breakdown of an configuration item which has no direct impact for a service is an incident. Note that incidents need to be treated and prioritized according to \emph{impact} and \emph{urgency}. Then, a problem is a not known cause for one or several incidents. There are \emph{problem records} with the complete history of a problem and also a \emph{Known Error Data Base} (KEDB) with all of these known errors and their records.

The single point of access for the customer is always the \emph{service desk}. Each call of a customer is an incident.

Consider different events, incidents and problems for your email service and check how they can be formally treated with your software system. How is the work of the service desk supported? Treat events, incidents and problems based on a few examples. 

\chapter{Conclusion}

Conclude your evaluation with some overall statements (about 1/2 page).