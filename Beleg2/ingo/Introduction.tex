Alle Ma�nahmen und Methoden f�r die Verwaltung von IT-Services lassen sich unter dem Begriff IT Service Management zusammenfassen. Ziel von IT Service Management ist es, IT-Services auszuliefern, die die Gesch�ftsprozesse effizient unterst�tzen. Mittels IT Service Management k�nnen Verbesserungen in den Service Management Prozessen identifiziert und implementiert werden. IT Services durchlaufen verschiedene Phasen in ihrem Lebenszyklus. 
In ITIL, der IT-Infrastructure Library, einem Good Practice f�r IT Service Management, werden folgende Phasen unterschieden:
\begin{itemize}
 \item Service Strategy
 \item Service Desig
 \item Service Transition
 \item Service Operation
 \item Continual Service Improvement
\end{itemize}

Jeder dieser Phasen im Service-Lifecycle besteht aus ITIL Prozessen. Jede Phase hat ein bestimmtes Ziel.

In der Service Strategy Phase wird eine Strategie definiert, wie der IT-Service f�r den Kunden bereitgestellt werden kann. Hier wird festgelegt, welche Services die IT-Organisation anbietet um sich strategisch auszurichten.

In der Service Desig Phase werden u.a. neue IT Services entworfen, bzw. vorhandene Services verbessert.

In der Service Transition Phase werden IT-Services ausgerollt auch unter Ber�cksichtigung von �nderungen an bestimmten IT-Services.

In der Service Operation Phase wird f�r eine effektive und effiziente Erbringung der IT-Services gesorgt, indem z.B. L�sungen f�r ein auftretendes Problem gesucht werden.

In der Phase Continual Service Improvement wird die Effektivit�t und die Effizienz von IT-Services fortlaufend verbessert, indem z.B. Methoden des Qualit�tsmanagements eingesetzt werden.

Software wie Nagios XI unterst�tzen die Ziele, die z.B. in der Phase Service Operation verfolgt werden, indem es Netzwerke, Hosts und Dienste �berwachen kann. Dabei unterst�tzt Nagios z.B. den ITIL-Prozess Event Management, indem es Configuration Items und IT-Services rund um die Uhr �berwacht und dabei Events wirft und filtert und kategorisiert, damit geeignete Ma�nahmen eingeleitet werden k�nnen.