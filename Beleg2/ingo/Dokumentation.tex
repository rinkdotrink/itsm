F�r Nagios XI gibt es viele Tutorials, die u.a. die Bereiche Installation, Konfiguration und Benutzung von Nagios XI abdecken.

Des weiteren gibt es eine technische Dokumentation um z.B. seine Nagios XI Installation besser kennenzulernen und zu verwalten.

Es gibt Anleitungen sowohl f�r Benutzer als auch f�r Administrator von Nagios XI.

Es gibt unter 

\url{http://library.nagios.com/}

eine Nagios Library, die offizielle Tipps, Tutorials, Videos, Dokumentationen und Anleitungen �ber Nagios Core, Nagios XI und andere Nagios Projekte enth�lt.

Es gibt FAQs zu Nagios Core unter:
\url{http://support.nagios.com/knowledge-base/faq}

Es gibt eine Nagios Wiki unter:
\url{http://wiki.nagios.org/}

zu dem auch viel die Nagios Community beitr�gt.

Und es gibt einen Nagios Quickstart Installation Guide unter:
\url{http://nagios.sourceforge.net/docs/3_0/quickstart.html}

Die Dokumentation ist au�ergew�hnlich umfangreich. Es werden 5 Punkte vergeben.

Es gibt verschiedene Arten von Trainings. 

Zum einen gibt es ein kostenpflichtiges Live-Training, indem die Teilnehmer Fragen stellen k�nnen. Es handelt sich dabei um ein online-Training. Die Preise f�r die verschiedenen Live-Trainings sind in folgender Tabelle angegeben:

\begin{tabular}{l|l}
Training & Course	Price\\
\hline
Basic Nagios Training Course	&	\$995 USD \\
Advanced Nagios Training Course &	\$1,495 USD \\
Linux Command Line Training Course &	\$295 USD \\
\end{tabular}

Zum anderen gibt es ein kostenpflichtiges Selbst-Training, indem sich der Teilnehmer mittels Online-Training-Material selbst weiterbilden kann.

Das Selbst-Training kostet \$199 USD f�r 12 Monate. Der Zugriff auf das Selbst-Training ist, f�r Kunden mit einem Nagios XI oder einer Nagios Core Support- und Wartungs-Vertrag, frei.

Ein authorisierter Partner von Nagios Trainings, der Nagios Trainingskurse anbietet, ist z.B. CyberMontana, zu finden unter:

\url{http://www.spidertools.com/}