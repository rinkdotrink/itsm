%%======================================================================
%% Hauptdokument
%%======================================================================

\documentclass[
    11pt,               % KOMA default
    a4paper,            % DIN A4
    twoside,            % Zweiseitig
    headsepline,        % Linie unter der Kopfzeile
    foodsepline,        % Linie �ber Fussnote
    automark,           % Kolumnentitel lebendig
    smallheadings,      % �berschriften eher klein setzen
    pointlessnumbers,   % Keinen Punkt hinter die letzte Zahl
                        % eines Kapitels (auch bei Anhang)
    openleft,           %
    cleardoubleplain,   %
    abstracton,         %
    idxtotoc,           % Index soll im Inhaltsverzeichnis auftauchen
    liststotoc,         %
    bibtotoc,           %
    parskip,            % parskip-, parskip*, parskip+
    DIVclassic,
]{scrreprt}

\pdfcompresslevel=9     % compression level for text and image;

\usepackage[latin1]{inputenc}
                        % Eingabe von �,�,�,� erlaubt
\usepackage[thmmarks]{ntheorem}
\usepackage[OT2,T1]{fontenc}
\usepackage{ae,aecompl}
\usepackage{ngerman,    % neue deutsche Rechtschreibung
    calc,               % Erweiterung der arithmetischen Funktionen in LaTeX
    color,              %
    epigraph,           % Zitat am Kapitelanfang
    fancybox,           % shadowbox, doublebox, ovalbox, Ovalbox
    fancyvrb,           % verbatim Erweiterung:
    float,              %
    mdwlist,            % compact list: itemize* ..
    scrdate,            % \todaysname
    scrtime,            % \thistime
    scrpage2,           % Kopf- und Fu�zeilen flexibel gestalten
    verbatim,           % Darstellung von "Text, wie er eingegeben wird"
    tabularx}           % Blocksatzspalten
\usepackage{listings}
\usepackage[noend]{algorithmic}
\usepackage[chapter]{algorithm}
\usepackage{subfigure}
\usepackage{textcomp}
\usepackage{thumbpdf}
\usepackage[pdftex]{graphicx}
\usepackage{times}
\usepackage[bookmarksopen,colorlinks,linkcolor=black,urlcolor=black,citecolor=black]{hyperref}
\usepackage{dsfont}
\usepackage{amsmath}
\usepackage{amssymb}
\usepackage{amstext}
\usepackage{soul}
\usepackage{psfrag}
\usepackage[russian,german,ngerman,USenglish]{babel}
\usepackage{listings}
\usepackage{paralist}
\usepackage{enumerate}
\usepackage{rotating}
\usepackage{multirow}
\usepackage{pdflscape}


\definecolor{orange}{rgb}{1.0,0.8,0.4}
\definecolor{blue}{rgb}{0.2,0.2,0.6}
\definecolor{black}{rgb}{0.0,0.0,0.0}
\definecolor{lightgray}{cmyk}{0.0,0.0,0.0,0.06}
\definecolor{darkgray}{cmyk}{0.0,0.0,0.0,0.5}

\setlength{\topmargin}{-0.5cm}
\setlength{\oddsidemargin}{0mm}
\setlength{\evensidemargin}{-5mm}
\setlength{\textwidth}{164mm}
\setlength{\textheight}{228mm}

\setlength{\parskip}{1.5ex plus0.5ex minus0.5ex}
\setlength{\parindent}{0cm}\renewcommand{\baselinestretch}{1.0}

\pagestyle{scrheadings} % Standart  Kopf- und Fu�zeile
\setkomafont{pagehead}{\small\scshape}

\setcounter{tocdepth}{3}

\fussy                  % viele Worttrennungen, "sch�nere" Wortabst�nde

\graphicspath{{../img/}}
\usepackage{makeidx}
\renewcommand{\indexname}{GLOSSARY}

\makeindex

\newcommand{\mylist}{
  \vspace{-0.2cm}
  \begin{list}{$\circ$}{
   %\setlength{\itemindent}{-0.2cm}
   \setlength{\itemsep}{0pt}
   \setlength{\parsep}{0pt}
   \setlength{\topsep}{0pt}
   \setlength{\partopsep}{0pt}
   \setlength{\leftmargin}{0.6em}
   \setlength{\labelwidth}{1em}
   \setlength{\labelsep}{0.5em}
  }
}
\newcommand{\mylistend}{
  \end{list}
  \vspace{-0.3cm}
  $\;$
}


\newtheorem{Def}{\sc{Definition}}[chapter]
\newtheorem{The}[Def]{\sc{Theorem}}
\newtheorem{Lem}[Def]{\sc{Lemma}}
\newtheorem{Cor}[Def]{\sc{Corollary}}
\newtheorem{Fac}[Def]{\sc{Fact}}
\newtheorem{MT}[Def]{\sc{Main Theorem}}
\newtheorem{Con}[Def]{\sc{Conjecture}}

\theorembodyfont{\rm}
\theoremstyle{nonumberplain}
\theoremsymbol{\ensuremath{\Diamond}}\theoremseparator{:}
\theorembodyfont{\rm}
\newtheorem{Exa}{\sc{Example}}
\newtheorem{Rem}{\sc{Remark}}

\theoremsymbol{\ensuremath{\Box}} \theoremseparator{.}
\theorembodyfont{\rm}
\newtheorem{Pro}{\sc{Proof}}

\renewcommand{\algorithmicrequire}{\textbf{Input:}}
\renewcommand{\algorithmicensure}{\textbf{Output:}}
\renewcommand{\listalgorithmname}{LIST OF ALGORITHMS}
\renewcommand{\listtablename}{LIST OF TABLES}
\renewcommand{\listfigurename}{LIST OF FIGURES}
\newcommand{\theHalgorithm}{\arabic{algorithm}}

\renewcommand{\labelitemi}{$\circ$}
\renewcommand{\labelitemii}{\mathversion{bold}$\cdot$\mathversion{normal}}
\newcommand{\bs}[1]{\boldsymbol{#1}}

\floatname{algorithm}{ALGORITHM}

\newcommand{\alg}[1]{\footnotesize\textsf{#1}\normalsize}
\newcommand{\udl}[1]{\underline{#1}}
\newcommand{\ve}[1]{\textbf{#1}}
\newcommand{\ma}[1]{\textbf{#1}}
\newcommand{\abs}[1]{\ensuremath{\left\vert#1\right\vert}}
\newcommand{\ggt}[2]{\mbox{\textrm{ggT}$(#1,#2)$}}
\newcommand{\ointa}[1]{\ensuremath{\oint\hspace{-0.3cm}\int_{\hspace{-0.1cm}
#1}}}
\newcommand{\inta}[1]{\ensuremath{\int\hspace{-0.3cm}\int_{\hspace{-0.1cm}
#1}}}
\newcommand{\ucirc}[1]{\ensuremath{\stackrel{\;\circ}{#1}}}
\newcommand{\udcirc}[1]{\ensuremath{\stackrel{\circ\circ}{#1}}}
\newcommand{\lrline}[0]{\begin{tabular}{l} \hspace{14.4cm}$\;$\\ \hline
\end{tabular}}
\newcommand{\bracket}[4]{\ensuremath{\left(\frac{\partial #1}{\partial
#2}#3\right)_{#4}}}
\newcommand{\ignore}[1]{\ensuremath{}}
\newcommand{\nab}[2]{\ensuremath{\frac{\partial ^#1}{\partial #2^#1}}}
\newcommand{\bra}[1]{\ensuremath{\langle #1|}}
\newcommand{\ket}[1]{\ensuremath{|#1\rangle}}
\newcommand{\braket}[2]{\ensuremath{\langle #1|#2\rangle}}
\newcommand{\braketH}[3]{\ensuremath{\langle #1|#2|#3\rangle}}
\newcommand{\na}[1]{\textsc{#1}}

\includeonly{
  metadaten,
  titel,
  inhalt,
  anhang,
  verzeichnisse
}

\begin{document}
%%======================================================================
%% Metadaten
%%======================================================================
\newcommand{\dcsubject}{Beleg}
\newcommand{\dctitle}{Evaluieren von Nagios XI}
\newcommand{\dcsubtitle}{~} % Untertitel, falls erforderlich

\newcommand{\dcauthorsurname}{K�rner}
\newcommand{\dcauthorname}{Ingo}
\newcommand{\dcauthoremail}{siinkoer@hszg.de}
\newcommand{\dcdate}{\today}

\newcommand{\dcplace}{G�rlitz}
\newcommand{\dcuni}{University of Applied Sciences Zittau/G�rlitz}
\newcommand{\dcdepart}{Fakult�t Elektrotechnik und Informatik} % Fakult�tsangabe
\newcommand{\dcprof}{} % Angabe der Professur

\newcommand{\dcpruefer}{Prof. Dr. L�ssig}% Pr�fer der Arbeit
\newcommand{\dcadvisor}{}% Zeitbetreuer der Arbeit


\newcommand{\dckeywords}{Nagios XI, Evaluierung}
\newcommand{\dcdateofbirth}{29.03.1983}
\newcommand{\dcplaceofbirth}{ Gro�enhain}
\newcommand{\dcmatrikel}{40586}

%%======================================================================
\hypersetup{%
    pdftitle    = {\dctitle}, %
    pdfsubject  = {\dcsubject, \dcdate}, %
    pdfauthor   = {\dcauthorname~\dcauthorsurname, \dcauthoremail}, %
    pdfkeywords = {\dckeywords}, %
    pdfcreator  = {pdfTeX with Hyperref and Thumbpdf}, %
    pdfproducer = {LaTeX, hyperref, thumbpdf}, %
}
%%======================================================================

%%======================================================================
%% Schmutztitel
%%======================================================================

\extratitle{
   \thispagestyle{empty}
   \usekomafont{sectioning}\mdseries
   \begin{center}
       \Huge \dcsubject\\[1.5ex]
       \hrule
       \vspace*{\fill}
       \includegraphics[width=14cm]{Images/HSZiGr.jpg}
   \end{center}
   \cleardoubleemptypage
}


%%======================================================================
%% Titelkopf
%%======================================================================
\titlehead{
    \vspace*{-1.5cm}
    \usekomafont{sectioning}\mdseries
    \begin{center}
        \raisebox{-1ex}{\includegraphics[width=14cm]{Images/HSZiGr.jpg}}\\
        \hrulefill \\[1em]
        {\Large\dcdepart}\\[0.5em]
        \dcprof
    \end{center}
    \vspace*{3.5cm}
}

%%======================================================================
%% Subjekt
%%======================================================================
\subject{\bf\huge\dcsubject}


%%======================================================================
%% Titel
%%======================================================================
\title{\sf\Huge
    \dctitle
    \\
    \dcsubtitle
    \\\vspace{2.3cm}
}

%%======================================================================
%% Autor des Dokumentes
%%======================================================================
\author{\dcauthorname~\dcauthorsurname,$\;$
\\\vspace{1cm}}

%%======================================================================
%% Ort, Datum
%%======================================================================
\date{\dcplace, DD$^{\text{th}}$ Month 20JJ}

%%======================================================================
%% bibliografische Angaben
%%======================================================================

\uppertitleback{\textbf{Bibliographic Description}\\\\
    \textsc{\dcauthorsurname, \dcauthorname},\\
    \textbf{\dctitle,}
    \dcsubtitle\\
    University of Applied Sciences Zittau/G�rlitz, Faculty of ...\\
    AssignmentType, 20JJ\\
    -PP pages, -FF figures, -TT tables, -CC citations, -AA algorithms}

\lowertitleback{
    \begin{tabular}{lrcl}
    \hspace{1cm}&submitted by:&\hspace{0.5cm}&\dcauthorname$\;$\dcauthorsurname\\
    &&&born on DD$^{\text{th}}$ Month JJJJ in\dcplaceofbirth\\
    &&&\dcauthoremail\\
    &&&\\
    &submitted on:&&DD$^{\text{th}}$ Month 20JJ\\
    &&&\\
    &Advisors:&& \dcpruefer$^{\;1}$\\
    &&&\dcadvisor $^{\;2}$\\
    &&&\\
    &Defence:&&prospectively DD$^{\text{th}}$ Month 20JJ
\end{tabular}
\\
\\
\\
\\
\\
\\
{\footnotesize
$\;^1$ University of Applied Sciences Zittau/G�rlitz, Faculty of ...\\
$\;^2$ University of Applied Sciences Zittau/G�rlitz, Faculty of ...}
}

%%======================================================================
%% maketitle
%%======================================================================

\maketitle

%%======================================================================
%%      Kurzfassung / Abstract in Englisch
%%======================================================================

\renewcommand{\baselinestretch}{1.2}\normalsize
\vspace{-3cm}
\thispagestyle{empty}
$\;$\vspace{-0.5cm}
\begin{center}
\textbf{Abstract}
\end{center}
\begin{quote}
\underline{$\;$\hspace{14.3cm}$\;$}\\

\vspace{-0.5cm}

In the ...
\\

\textbf{Keywords:} Keyword 1, Keyword 2, Keyword 3\\
\underline{$\;$\hspace{14.3cm}$\;$}
\end{quote}
\cleardoubleemptypage

\renewcommand{\baselinestretch}{1.0}\normalsize

%%======================================================================
%%      Kurzfassung / Abstract in Deutsch
%%======================================================================

\selectlanguage{ngerman}

\renewcommand{\baselinestretch}{1.2}\normalsize

\thispagestyle{empty}
\def\abstractname{Referat}
$\;$\vspace{-1.3cm}
\begin{center}
\textbf{Referat}
\end{center}
\begin{quote}
\underline{$\;$\hspace{14.3cm}$\;$}\\

\vspace{-0.5cm}

In der ...
\\

\textbf{Schlagworte:} Schlagwort 1, Schlagwort 2, Schlagwort 3\\
\underline{$\;$\hspace{14.3cm}$\;$}
\end{quote}
\cleardoubleemptypage

\selectlanguage{USenglish}

\renewcommand{\baselinestretch}{1.0}\normalsize

%%======================================================================
%%      Inhaltsverzeichnis
%%======================================================================
\cleardoubleemptypage
\pagenumbering{Roman}
\renewcommand{\contentsname}{TABLE OF CONTENTS}

\tableofcontents

%%======================================================================
%% Haupttext
%%======================================================================

\renewcommand{\baselinestretch}{1.2}\normalsize

\chapter{INTRODUCTION}
\label{sec:introduction}

\pagenumbering{arabic}
%\setcounter{page}{2}
\setcounter{footnote}{1}

Write something about IT Service Management, its goals and why software supports achieving these goals. Also describe goals of your project here - why are we doing this? (About one page in sum.)

\chapter{Evaluation of General Characteristics of the Software}

Write a few introductory lines.

\section{Basis facts about the software solution}
Please describe the basic characteristics of the software. 

Is is a web-based solution?

Ja

Is it installable or just available as a service? 

Die Software ist installierbar.



How much effort does the installation take (describe and give a time for the installation)?

Die folgenden Schritte werden auf folgendem Testrechner ausgef�hrt:

Ubuntu 12.04

Es wird die Free Trial Version von Nagios XI heruntergeladen unter:

http://www.nagios.com/products/nagiosxi

Die Trial Version ist eine 60 Tage Testversion von Nagios XI.

Es wird die VMware Virtual Machine (64-bit) Version 2012R1.6 heruntergeladen unter:

http://library.nagios.com/library/products/nagiosxi/downloads/main

Auf der viruellen Maschine l�uft CentOS 6.x und Nagios XI 2012 ist installiert.

Um die VMware Virtual Machine starten zu k�nnen, muss VMware installiert werden.
VMware kann kostenfrei heruntergeladen werden unter:

\url{https://my.vmware.com/web/vmware/free#desktop_end_user_computing/vmware_player/5_0}

Der heruntergeladene VMware-Player wird installiert mit:

sudo sh VMware-Player-5.0.1-894247.x86\_64.bundle

Der VMware-Player wird gestartet und es wird die virtuelle Maschine ausgew�hlt mit 

Open a virtual Maschine.

Die ausgew�hlte virtuelle Maschine wird gestartet mit 

play virtual maschine.

Die virtuelle Maschine wird gestartet. CentOS wird gebootet und Nagios gestartet.

Auf Nagios XI kann jetzt im Browser zugegriffen werde unter:

http://192.168.0.100/

\begin{figure}[htp]
\centering
\includegraphics[width=0.6\textwidth]{ingo/bilder/Startseite}
\caption{Startseite von Nagios}
\label{fig:StartseiteVonNagios}
\end{figure}

Das Default Root Passwort ist nagiosxi


How old is the software? When did its development start? 

1999 ver�ffentlichte Ethan Galstad Nagios - dass damals noch NetSaint hie� - als Open Source Projekt.
http://www.nagios.org/about/history

Which version is it (0.1?)? 

Netsaint 0.0.1
\url{http://www.ussrback.com/UNIX/audit/netsaint/index.html}

Is it well established? How many customers are using it? 

Es wird gesch�tzt, dass es weltweit ca 1 Million Nagios Nutzer gibt.
http://www.nagios.org/about/community

Is it still further developed - are new releases planned and when was the last release?

Es wird immernoch weiterentwickelt. Neue Releases sind geplant. Die letzte Version 2012R1.6 ist von 15. Februar 2013.

\section{Versions and price}
\input{ingo/VersionsAndPrice}

\section{Usability}
Das User Interface und das Layout wirken aufger�umt, funktional aber durch dicke Scroll-Leisten nicht ganz up to date. Es werden hierf�r 3 Punkte vergeben.

Die Anwendung ist leicht zu verstehen und die Funktionalit�ten der Software sind leicht zu erlernen. Das Userinterface wirkt effizient.

Die Anwendung ist �ber eine Art Registerreiter in folgende Bereiche gegliedert:
\begin{itemize}
 \item Home
\item Views
\item Dashboards
\item Reports
\item Configure
\item Tools
\item Help
\item Admin
\end{itemize}

F�r jeden Registerreiter k�nnen die anzuzeigenden Informationen in einer Sidebar verfeinert werden.
Um die Verfeinerungen schneller �berblicken zu k�nnen, sind sie durch �berschriften gruppiert.

Die Funktionalit�ten der Anwendung sind somit au�ergew�hnlich gut zu verstehen und zu lernen. Es werden hierf�r 5 Punkte vergeben.

Die Funktionalit�ten intuitiv, da sie klar benannt sind. Unter dem Admin Bereich k�nnen typische Verwaltungseinstellungen gesetzt werden f�r z.B. Nutzer oder das System selbst. Es werden f�nf Punkte vergeben.

Mit Farben wurde an den Stellen gearbeitet, wo Informationen besonders hervorgehoben werden sollen. Die Farbwahl ist au�ergew�hnlich gut. Es werden 5 Punkte vergeben.

\section{Performance}
Describe if the performance of the software. Is it satisfying? Are the response times in order?

Unter 

http://nagiosxi.demos.nagios.com/

kann sich auf ein Nagios XI Demo System eingeloggt werden.

Die Performance der Andwendung ist au�ergew�hnlich schnell, d.h. 5.
Sie l�dt, wie bei einer AJAX-Anwendung, nur die Teile nach, die sich ver�ndern.

Are there any performance critical operations? 
Es wurden keine performancekritischen Operationen entdeckt.

Please answer the questions and evaluate the performance between 5-extraordinary good and 1-the screen war frozen with the first klick :)


\section{Documentation}
F�r Nagios XI gibt es viele Tutorials, die u.a. die Bereiche Installation, Konfiguration und Benutzung von Nagios XI abdecken.

Des weiteren gibt es eine technische Dokumentation um z.B. seine Nagios XI Installation besser kennenzulernen und zu verwalten.

Es gibt Anleitungen sowohl f�r Benutzer als auch f�r Administrator von Nagios XI.

Es gibt unter 

\url{http://library.nagios.com/}

eine Nagios Library, die offizielle Tipps, Tutorials, Videos, Dokumentationen und Anleitungen �ber Nagios Core, Nagios XI und andere Nagios Projekte enth�lt.

Es gibt FAQs zu Nagios Core unter:
\url{http://support.nagios.com/knowledge-base/faq}

Es gibt eine Nagios Wiki unter:
\url{http://wiki.nagios.org/}

zu dem auch viel die Nagios Community beitr�gt.

Und es gibt einen Nagios Quickstart Installation Guide unter:
\url{http://nagios.sourceforge.net/docs/3_0/quickstart.html}

Die Dokumentation ist au�ergew�hnlich umfangreich. Es werden 5 Punkte vergeben.

Es gibt verschiedene Arten von Trainings. 

Zum einen gibt es ein kostenpflichtiges Live-Training, indem die Teilnehmer Fragen stellen k�nnen. Es handelt sich dabei um ein online-Training. Die Preise f�r die verschiedenen Live-Trainings sind in folgender Tabelle angegeben:

\begin{tabular}{l|l}
Training & Course	Price\\
\hline
Basic Nagios Training Course	&	\$995 USD \\
Advanced Nagios Training Course &	\$1,495 USD \\
Linux Command Line Training Course &	\$295 USD \\
\end{tabular}

Zum anderen gibt es ein kostenpflichtiges Selbst-Training, indem sich der Teilnehmer mittels Online-Training-Material selbst weiterbilden kann.

Das Selbst-Training kostet \$199 USD f�r 12 Monate. Der Zugriff auf das Selbst-Training ist, f�r Kunden mit einem Nagios XI oder einer Nagios Core Support- und Wartungs-Vertrag, frei.

Ein authorisierter Partner von Nagios Trainings, der Nagios Trainingskurse anbietet, ist z.B. CyberMontana, zu finden unter:

\url{http://www.spidertools.com/}

\section{Support}
Mit einer Nagios XI Lizenz enth�lt:

\begin{itemize}
 \item technischen Support - via EMail oder in einem speziellen customer-only Abschnitt im online forum unter

\url{http://support.nagios.com/forum}

Je nach Lizenz-Level werden pro Jahr bis zu 10 Support Incidents bearbeitet.

\item Training - ein volles Jahr Zugriff auf Ressourcen f�r das Selbststudium von Nagios XI und Nagios addons.

\item eine unbefristete Lizenz - die Lizenz der Software ist dauerhaft, selbst wenn kein zuk�nftiger Support oder Wartungsvertr�ge geschlossen werden.

\item die Nagios Library - f�r ein volles Jahr kann auf spezielle Nagios Libraries zugegriffen werden, mit customer-only Tutorials, Videos und Tech Tipps.

\item Produkt-Einfluss - Feature Requests k�nnen eingereicht werden.

\item freie Lizenzwahl f�r selbstgeschriebene Erweiterungen - es k�nnen beliebige Lizenzen f�r eigene Erweiterungen f�r Nagios XI gew�hlt werden: z.B. open source, propriet�r oder public domain.
\end{itemize}

Die Leistungen und die Preise f�r den Support richten sich nach dem Lizenz-Level, dass der Kunde gekauft hat. Die Lizenzlevel sind in der Tabelle Lizenzleel auf Seite \pageref{LizenzLevel} aufgef�hrt.

Ein Eintrag im Forum wird innerhalb von 24 Stunden beantwortet.


\section{Errors and Bugs}
Nach der Anleitung in den Video-Tutorials, speziell nach dem nagios-xi-web-interface-setup-guide unter

\url{http://library.nagios.com/library/products/nagiosxi/tutorials/}

war kein Zugriff mehr auf Nagios �ber das Web Interface m�glich. Es kommt folgende Fehlermeldung, wie in Abbildung \ref{fig:NagiosXiFehlermeldung} auf Seite \pageref{fig:NagiosXiFehlermeldung} zu sehen. �ber das Web-Interface kann nicht mehr auf Nagios zugegriffen werden.

Auch nachdem die virtuelle Maschine gel�scht wurde und durch eine neu heruntergeladene ersetzt wurde, konnte nicht auf das Web-Interface zugegriffen werden.

\chapter{Evaluation of the ITSM Specific Functionalities}

\section{ITSM Processes}
Describe the support of the processes with a number from 1-5 (5 for fully supported), describe also "what" it supported and "what not". Use our scribe notes with detailed information about the processes to have an idea about what could be supported. Give further descriptions "how" it is supported and about the usability of the process features in the column "comments". Describe as many details as possible.

\begin{table}[h!]
\caption{Supported ITSM Processes}
\vspace*{0.3cm}
\begin{tabular}{|p{6.1cm}|p{4.5cm}|p{4.5cm}|}\hline
\textbf{ITSM Process}             &\textbf{Supported (1-5)}        &\textbf{Comments}\\\hline\hline
\textbf{Service Strategy}          &      1                          &\\
Strategy Generation            &                                &\\
Demand Management                   &                                &\\
Service Portfolio Mgmt.                   &                                &\\
Financial Management                   &                                &\\\hline
\textbf{Service Design}          &                                &\\
Service Catalogue Mgmt.                   &                                &\\
Service Level Management                   &                                &\\
Capacity Management                   &    2                            &\\
Availability Management                   &    2                            &\\
IT Service Continuity Mgmt.                   &                                &\\
Information Security Mgmt.                   &                                &\\
Supplier Management                   &                                &\\\hline
\textbf{Service Transition}          &      1                          &\\
Transition Planning and Support                   &                                &\\
Change Management                   &                                &\\
Service Asset and Configuration Mgmt.                   &                                &\\
Release and Deployment Mgmt.                   &                                &\\
Service Validation and Testing                   &                                &\\
Evaluation                   &                                &\\
Knowledge Management                   &                                &\\\hline
\textbf{Service Operation}          &                                &\\
Event Management                   &         4                       &\\
Incident Management                   &        2                        &\\
Request Management                   &          1                      &\\
Problem Management                   &         2                       &\\
Access Management                   &           1                     &\\\hline
\textbf{Continual Service Improvement}          &   1                             &\\
7-Step Improvement                   &                                &\\
Service Reporting                   &                                &\\
Service Measurement                 &                                &\\\hline
\textbf{General}                   &           1                      &\\
Service Desk                       &                                 &\\
Raci Authority Matrix                    &                           &\\\hline
\end{tabular}
\label{tab:SupportedITSMProcesses}
\end{table}



\section{IT Service Management Roles}
\subsection{Authorization Levels}
\begin{enumerate}
\item Admin \\
Users that are configured with an Authorization Level of Admin will have the ability to access, add, and re-configure:
\begin{itemize}
\item Users
\item Hosts
\item Services
\item Components
\item Configuration Wizards
\item Dashlets
\item Program Settings
\item Security Credentials
\end{itemize}
\item User
\\
Users with an Authorization Level of User will only be able to see and modify hosts and services for which they are a notification contact.
\end{enumerate}
\subsection{User Security Settings}
Administrators may grant users additional rights beyond the default permissions they have in Nagios XI.
\begin{itemize}
\item Can see all hosts and services \\
The user can see all hosts and services that are being monitored.
\item Can (re)configure hosts and services \\
The user can re-configure and delete existing hosts and services(which they can see) from the monitoring configuration, and add new hosts and services to the monitoring configuration.
\item Can control all hosts and services \\
The user can see and submit commands(e.g. disable notifications) for hosts and services.
\item Can see/control monitoring engine \\
The user can see and control(e.g.shutdown or restart) the monitoring engine.
\item Can access advanced features \\
The user can see the Advanced tab when viewing the details of hosts and services, can see some menu items and options that are not usually suitable for basic users. 
\item Has read-only access \\
It removes the user's ability to submit commands(e.g. diable notifications) for hosts and services. It also prevents them from re-configuring existing or adding new hosts and services.
\end{itemize}
\begin{figure}[htp]
\centering
\includegraphics[width=0.6\textwidth]{Christof/Bilder/Security}
\caption{Security Settings}
\label{fig:SecuritySettings}
\end{figure}

\section{Scenarios}
\subsection{Design of an Email Service}
Nagios ist ein Netzwerk�berwachungstool, das die Aufgabe hat, die Administratoren schnell �ber bedenkliche(WARNING) oder kritische Zust�nde(CRITICAL) zu benachrichtigen. Die Benachrichtigung erfolgt per E-Mail oder SMS. Der Administrator legt in der Konfiguration fest welche Zust�nde "bedenklich" und welche schon "kritisch" sind. Es k�nnen s�mtliche Ressourcen eines Netzwerks, wie beispielsweise Router, Server, Switches und Dienste �berwacht werden. Es werden in regelm��igen Abst�nden Plugins aufgerufen, die den Zustand eines Services pr�fen. Nagios unterst�tzt haupts�chlich den Event-Management, und in der Design Phase werden die beiden Prozesse "Capacity Management" und "Availability Management" nur teilweise abgedeckt. Es k�nnen keine neuen Services, wie z.B. Email-Service, entwickelt werden.

\subsection{ITIL Roles}
Nagios unterst�tzt keine Rollen im Sinne von ITIL.


\subsection{Service Catalogue}
\label{sec:serviceCatalogue}
Nagios verf�gt �ber keinen Service Catalogue, der Informationen zu allen in Betrieb befindlichen und geplanten Services enth�lt.

\subsection{Further Management of the Email Service}
\label{sec:furtherManagement}

Nagios hat die Aufgabe, den Administrator schnell und gezielt �ber Fehlerzust�nde zu informieren. Der Administrator soll dabei nicht mit Informationen �ber fehlerfrei laufenden Dienste und Hosts �berflutet werden. Die �berwachung wird bei Nagios mit Plugins durchgef�hrt, die lokal auf dem Nagios-Server oder auch auf den zu �berwachenden Hosts installiert werden k�nnen. Dadurch kann Nagios leicht um weitere Plugins erweitert werden. Die von Plugins gelieferten Performancedaten k�nnen ausgewertet und dargestellt werden.  

Es gibt keinen Capacity Management Information System und somit kann man mit Nagios auch keinen Capacity Plan erstellen. Die Rolle des Availability Managers als auch das Security Management und Supplier Management werden ebenfalls nicht unterst�tzt.

\subsection{Configuration Management}
\label{sec:configurationManagement}

Configuration Management wird nicht unterst�tzt.

\subsection{Change Management}
\label{sec:changeManagement}

Change Management wird nicht unterst�tzt.

\subsection{Definite Media Library}
\label{sec:definiteMediaLibrary}

Nagios enth�lt keine Definitive Media Library, in der die Medien-CIs aufbewahrt werden und abgesichert werden.

\subsection{Knowledge Management}
\label{sec:knowledgeManagement}

Nagios unterst�tzt kein Knowledge Management.

\subsection{Event, Incident and Problem Management}
\label{sec:eventIncidentProblemManagement}

Nagios unterscheidet zwischen Ereignissen und Host-und Service-Zust�nden:
\begin{enumerate}
\item Host-Check \\
Ein Host-Check pr�ft ob ein Rechner erreichbar ist. Dabei kommt h�ufig ein einfaches \textit{ping} zum Einsatz. Host-Checks werden haupts�chlich durchgef�hrt, wenn auf dem Rechner alle laufenden Netzwerkdienste nicht erreichbar sind.
\item Service-Check \\
Bei einem Service-Check werden einzelne Netzwerkdienste �berpr�ft, wie beispielsweise HTTP, SMTP etc. aber auch laufende Prozesse, Logfiles oder die CPU-Last.
\item Ereignisse \\
Ein Ereignis taucht in Form eines Syslog-Eintrags oder eines SNMP-Traps auf. SNMP-Traps sind asynchrone Nachrichten, die ein SNMP-Agent(Dienst) an eine zentrale Management-Einheit versendet. Bei SNMP handelt es sich um ein Protokoll, welches die Kommunikation zwischen so genannten Managern und Agenten definiert. Der Manager ist die Arbeitskonsole des Administrators, w�hrend die Agenten direkt auf den Systemen und Netzwerkkomponenten laufen, die �berwacht oder konfiguriert werden sollen.
\\
Wenn im Syslog-Eintrag die Meldung erscheint, dass ein Ger�t ausgefallen ist, �ndert sich der Eintrag erst wenn das Ger�t wieder funktionsf�hig ist. Wenn man also mit Logfile-Checks regelm��ig pr�ft, ob innerhalb der letzten halben Stunde ein Eintrag erfolgte, so liefert der Logfile-Check nach Ablauf einer halben Stunde ein OK, weil innerhalb der Zeit kein kritischer Ereignis aufgetreten ist. Tats�chlich aber befindet sich das Ger�t im kritischen Zustand.\\
�berwacht man das Ger�t per SNMP-Trap, so informiert ein Alarm-Trap �ber den Ausfall, und der Zustand des Services wird auf CRITICAL gesetzt bis wieder ein OK-Trap den Zustand auf OK setzt, nachdem das Ger�t wieder funktionsf�hig ist.
\\
Alle Events k�nnen in einer Event-Datenbank(EventDB) gesammelt werden. Die Events k�nnen �ber eine Webinterface oder per Nagios-Plugin abgefragt werden. Man kann �hnliche oder gleiche Ereignisse suchen. Der Administrator wird informiert ob zu einem Ereignis ein oder mehrere nicht best�tigte Eintr�ge vorliegen.
 
\end{enumerate}
F�r die Checks werden externe Programme(Plugins) verwendet. Nagios kann damit alles pr�fen was sich elektronisch messen l�sst. Lediglich m�ssen die Messdaten als eine vom Computer verwertbare Information bereitgestellt werden mit Hilfe von Sensoren und anderen Messger�ten.\\


\chapter{Conclusion}

Conclude your evaluation with some overall statements (about 1/2 page).
\cleardoublepage
\begin{appendix}
%%======================================================================
%% Anhang
%%======================================================================

\chapter{This is the First Appendix Chapter}
\label{sec:ThisIsTheFirstAppendixChapter}

\section{Appendix Section}
\label{sec:AppendixSection}

\chapter{This is the Second Appendix Chapter}
\label{sec:ThisIsTheSecondAppendixChapter}

\section{Another Appendix Section}
\label{sec:AnotherAppendixSection}

\end{appendix}
\pagenumbering{roman}
%\renewcommand{\listalgorithmname}{LIST OF ALGORITHMS}
\renewcommand{\listtablename}{LIST OF TABLES}
\renewcommand{\listfigurename}{LIST OF FIGURES}
\renewcommand{\indexname}{GLOSSARY}


%%======================================================================
%%      Abbildungsverzeichnis
%%======================================================================
\markboth{LIST OF FIGURES}{LIST OF FIGURES}
\listoffigures
\cleardoublepage

%%======================================================================
%%      Tabellenverzeichnis
%%======================================================================
\markboth{LIST OF TABLES}{LIST OF TABLES}
\listoftables

\cleardoublepage

%%======================================================================
%%      Algorithmenverzeichnis
%%======================================================================
\renewcommand{\listalgorithmname}{LIST OF ALGORITHMS}
\markboth{LIST OF ALGORITHMS}{LIST OF ALGORITHMS}
\listofalgorithms

\cleardoublepage

%%======================================================================
%%      Listingsverzeichnis
%%======================================================================
\renewcommand*\lstlistingname{LIST OF LISTINGS}
\renewcommand*\lstlistlistingname{LIST OF LISTINGS}
\lstlistoflistings
\cleardoublepage

%======================================================================
%  Abk�rzungsverzeichnis
%======================================================================
\twocolumn
\renewcommand{\baselinestretch}{1.0}\normalsize

\chapter*{LIST OF SYMBOLS}
\addcontentsline{toc}{chapter}{LIST OF SYMBOLS} \markboth{LIST OF SYMBOLS}{LIST OF SYMBOLS}

\section*{Notations}

\begin{tabular}{p{1.5cm}p{12.1cm}}
$\Box$ & end of a proof\\
\vspace{-0.22cm}$\Diamond$ & \vspace{-0.22cm}end of an example or a remark \\
\vspace{-0.22cm}$x$                                    & \vspace{-0.22cm}scalar \\
\vspace{-0.22cm}$\ve{x}$                               & \vspace{-0.22cm}vector \\
\vspace{-0.22cm}$\abs{\ve{x}}$                         & \vspace{-0.22cm}norm of vector $\ve{x}$, $\abs{\ve{x}}= \sqrt{\ve{x}\ve{x}}$\\
\vspace{-0.22cm}$\abs{\mathcal{S}}$                    & \vspace{-0.22cm}number of elements in a set $\mathcal{S}$\\
\vspace{-0.22cm}$\ve{x}\ve{y}$                         & \vspace{-0.22cm}scaler product of the vectors $\ve{x}$ and $\ve{y}$ \\
\vspace{-0.22cm}$\bs{X}$                               & \vspace{-0.22cm}matrix \\
\vspace{-0.22cm}$\bs{X}^{\mathrm{tr}}$                 & \vspace{-0.22cm}transposed matrix \\
\vspace{-0.22cm}\footnotesize\textsf{procedure}\normalsize() & \vspace{-0.22cm}some procedure in an algorithm\\
\vspace{-0.22cm}\footnotesize\textsf{\emph{data}}\normalsize & \vspace{-0.22cm}some data structure used in algorithms
\end{tabular}

\section*{Greek Symbols}

\begin{tabular}{p{1.5cm}p{5.1cm}}
$\alpha$ & state in the state space\\
\vspace{-0.22cm}$\bs{\alpha}$ & \vspace{-0.22cm}generalized state\\
\vspace{-0.22cm}$\beta$ & \vspace{-0.22cm}state in the state space\\
\vspace{-0.22cm}$\bs{\beta}$ & \vspace{-0.22cm}generalized state\\
\vspace{-0.22cm}$\Gamma^t_{\bs{\alpha}\bs{\beta}}$ & \vspace{-0.22cm}transition probability from population $\bs{\beta}$ to $\bs{\alpha}$ at time $t$
\end{tabular}

\section*{Latin Symbols}

\begin{tabular}{p{1.5cm}p{5.1cm}}
$a_i$ & a certain constant\\
\vspace{-0.22cm}$A(I)$ & \vspace{-0.22cm}algorithm $A$ started with instance $I$
\end{tabular}

\cleardoublepage

%======================================================================
%  Abk�rzungsverzeichnis
%======================================================================
\onecolumn
\renewcommand{\baselinestretch}{1.2}\normalsize

\chapter*{LIST OF ABBREVIATIONS}
\addcontentsline{toc}{chapter}{LIST OF ABBREVIATIONS}
\markboth{LIST OF ABBREVIATIONS}{LIST OF ABBREVIATIONS}
\vspace{-0.5cm}
\begin{tabular}{p{3cm}p{12.2cm}}
ACO & \udl{A}nt \udl{C}olony \udl{O}ptimization\\
AJAX & \udl{A}synchronous \udl{Ja}vaScript and \udl{X}ML\\
B2B & \udl{B}usiness-\udl{to}-\udl{B}usiness
\end{tabular}

\cleardoublepage

%======================================================================
%  Glossar
%======================================================================
\renewcommand{\baselinestretch}{1.1}\normalsize

\markboth{GLOSSARY}{GLOSSARY}
\printindex
\cleardoublepage

%======================================================================
%   Literaturverzeichnis
%======================================================================
\renewcommand{\baselinestretch}{1.13}\normalsize
\markboth{BIBLIOGRAPHY}{BIBLIOGRAPHY}
\renewcommand{\bibname}{BIBLIOGRAPHY}
\bibliographystyle{plain}
\bibliography{literatur}
\cleardoublepage

%%======================================================================
%%      Ende
%%======================================================================
\cleardoublepage
\pagenumbering{arabic}
\setcounter{page}{1}

%======================================================================
%   Selbstst�ndigkeitserkl�rung
%======================================================================
\selectlanguage{ngerman}
\chapter*{Selbstst�ndigkeitserkl�rung}
\thispagestyle{empty} Hiermit erkl�re ich, dass ich die vorliegende
Arbeit selbstst�ndig angefertigt, nicht anderweitig zu
Pr�fungszwecken vorgelegt und keine anderen als die angegebenen
Hilfsmittel verwendet habe. S�mtliche wissentlich verwendete
Textausschnitte, Zitate oder Inhalte anderer Verfasser
wurden ausdr�cklich als solche gekennzeichnet.\\[2ex]
\dcplace, den 25.02.2013\\[6ex]
\flushleft
\newlength\us
\settowidth{\us}{-\dcauthorsurname~\dcauthorname-}
\begin{tabular}{p{\us}}\hline
\centering\footnotesize \dcauthorsurname,$\,$\dcauthorname
\end{tabular}

\end{document}
